\documentclass{beamer}
\usetheme{Rochester}
\usecolortheme{seagull}
\usepackage{hyperref}
\setbeamercovered{transparent}

\setbeamertemplate{navigation symbols}{}
\hypersetup{colorlinks,linkcolor=,urlcolor=blue}

\title[]{An overview of the direct to consumer (DTC) genetics market}
\author[]{Vincent Plagnol}
\date{November 26 2012}
\subject{}
\institute{UCL Genetics Institute}


\AtBeginSection[] {
  \begin{frame}
    \frametitle{Outline}
    \tableofcontents[currentsection]
  \end{frame}
}




\begin{document}

\begin{frame}
  \titlepage
\end{frame}


\begin{frame}
 \frametitle{Outline}
 \tableofcontents
\end{frame}





%%%%%%%%%%%%%%%%%%%%%%%%%%%%%%%%%%%%%%%%%%%%%%%%%%%%%%%%%%%%%%%%%%%%%%%%%%%%%
\section{Defining the DTC genetics market and its legal issues}


\begin{frame}
  \frametitle{What does DTC mean?}
  \begin{itemize}
  \item Traditionally, genetic tests have been available only through health care providers such as physicians, nurse practitioners, and genetic counsellors. 
  \item Direct-to-consumer genetic testing refers to genetic tests that are marketed directly to consumers via television, print advertisements, or the Internet. 
    \begin{itemize}
    \item The test typically involves collecting a DNA sample at home, often by swabbing the inside of the cheek, and mailing the sample back to the laboratory.
    \item Consumers are notified of their results by mail or over the telephone, or the results are posted online
    \end{itemize}
  \end{itemize}
\end{frame}



\begin{frame}
  \frametitle{The direct-to-consumer genetic business}
   \begin{center}
     \includegraphics[width=8cm]{fig/23andMe.png}\\
     \vspace{1cm}
     \includegraphics[width=8cm]{fig/decodeme.png}\\
     \vspace{1cm}
     \includegraphics[width=8cm]{fig/lumigenix.png}\\
  \end{center}
\end{frame}



\begin{frame}
  \frametitle{What do they sell to you and at what price?}
   \begin{itemize}
   \item The main relevant product of interest is a genome-wide genotyping array, typically covering 500,000 markers.
     \begin{itemize}
     \item 23andMe: \$100 + \$10 per month.
     \item Lumigenix: \$479 one time cost.
     \item DecodeMe: \$1,100 one time cost.
     \end{itemize}
   \item Other companies and products exist in the ancestry market but they usually provide a different type of data (chromosome Y and mtDNA only)
   \end{itemize}
\end{frame}



\begin{frame}
  \frametitle{Chromosome Y and mtDNA data}
   \begin{itemize}
   \item Alternatively, there is a set of DTC companies that only focus on the chrY and mtDNA markers.
   \item See for example \href{http://www.britainsdna.com/}{Britain's DNA} or \href{http://www.familytreedna.com/}{FamilyTreeDNA}.
   \item Some clarifications on mtDNA and chrY DNA:
     \begin{itemize}
     \item If you go back just 12 generation - 300 years or so - you have around 4,000 ancestors, but mtDNA and chrY gives you information about only 2 of them.
     \item These are the fathers of fathers of fathers... and the mothers of mothers of mothers...
     \end{itemize}
   \item We will talk more about ancestry estimation later in the class.
   \end{itemize}
\end{frame}



\begin{frame}
  \frametitle{The difference between medical tests and DTC genetics}
  \begin{itemize}
  \item DTC genetics operates in a very grey area.
  \item Medical tests tend to be heavily regulated for reasons that are relatively obvious.
  \item But is getting some information about your DNA a medical test?
  \item Some US states and some European countries forbid the business of these companies.
    \begin{itemize}
    \item Germany has a very paternalistic approach for example: only a clinician should tell you something about your DNA.
    \end{itemize}
  \item Can you list arguments pro/against a deregulated market?
  \end{itemize}
\end{frame}




\begin{frame}
  \frametitle{Arguments pro-regulation}
  \begin{itemize}
  \item Patients do not know what to do with some of these results.
  \item Learning something about cancer risk may lead to panic which is unwarranted.
  \item Even if patients are knowledgable, these companies can make errors, switch samples...
    \begin{itemize}
      \item A 23andMe customer mistakenly learnt that her daughter was not her daughter (she in fact was but samples had been swapped).
    \end{itemize}
  \item Knowing something about your genome also impacts potentially your family members, so it is not just you.
  \item It is easy to put together a scam of some sort.
    \begin{itemize}
    \item Think pre-natal testing for example: get your money back if gender is predicted incorrectly.
    \end{itemize}
  \end{itemize}
\end{frame}



\begin{frame}
  \frametitle{Arguments against (heavy) regulation}
  \begin{itemize}
  \item Freedom: this is your genome, you have the right to know anything that is in it.
  \item Clinicians, or at least GP, do not know much more than you on the matter.
  \item There are potential health benefits to having that knowledge and we do not want to restrict this access.
  \item Regulations will slow down and perhaps kill a market that has the potential to offer valuable information.
  \end{itemize}
\end{frame}



\begin{frame}
  \frametitle{A discussion of Lumigenix's legal strategy}
  \begin{itemize}
  \item See this excellent \href{http://www.genomicslawreport.com/index.php/2011/06/16/dtc-genetic-testing-and-the-fda-is-there-an-end-in-sight-to-the-regulatory-uncertainty/}{post} from Dan Vorhaus on the legal uncertainties.
  \item Lumigenix, for its part, has thus far taken an intermediate approach, steering clear of reporting on variants with unambiguous clinical relevance while maintaining DTC access to its service.
  \item Lumigenix has opted to exclude certain information from its current service to avoid any clinical confusion.
  \item There are at least 196 sites on the chip that match the position and sequence of known Mendelian disease mutations, including 12 in the BRCA1/2 genes and 6 in the cystic fibrosis gene {\it CFTR}.
  \end{itemize}
\end{frame}

%%%%%%%%%%%%%%%%%%%%%%%%%%%%%%%%%%%%%%%%%%%%%%%%%%%%%%%%%%%%%%%%%%%%%%%%%%%%%
\section{Some online resources}


\begin{frame}
  \frametitle{Raw data are, or at least should be, available}
  \begin{itemize}
  \item The vast majority of these companies will let you access your raw data, typically with a flat text format.
  \item This is really useful as you can use these data with other tools, or potentially assess things yourself if you have the expertise.
  \item Some web browsers like OpenSNP will make it easy for you to browse your own data.
    \begin{itemize}
    \item See this \href{https://opensnp.org/snps/rs9939609}{link for rs9939609} for example.
    \end{itemize}
  \item An other resource, \href{http://www.snpedia.com/index.php/Promethease}{SNPedia} will build informative reports for you to read based on your raw data.
  \item It is however often hard to separate real information from background noise.
  \end{itemize}
\end{frame}


\begin{frame}
\frametitle{Genomes Unzipped}
  \begin{center}
    \includegraphics[width=12cm]{fig/GUNZIP.png}
  \end{center}
\end{frame}



\begin{frame}
\frametitle{The Genomes Unzipped team}
  \begin{center}
    \includegraphics[width=1.5cm]{fig/gunzip/daniel_final.jpg}
    \includegraphics[width=1.5cm]{fig/gunzip/luke_final.jpg}
    \includegraphics[width=1.5cm]{fig/gunzip/jeff_final.jpg}
    \includegraphics[width=1.5cm]{fig/gunzip/vince_final.jpg}\\
    \includegraphics[width=1.5cm]{fig/gunzip/dan_final.jpg}
    \includegraphics[width=1.5cm]{fig/gunzip/carl_final.jpg}
    \includegraphics[width=1.5cm]{fig/gunzip/caroline_final.jpg}
    \includegraphics[width=1.5cm]{fig/gunzip/ilana_final.jpg}\\
    \includegraphics[width=1.5cm]{fig/gunzip/don_final.jpg}
    \includegraphics[width=1.5cm]{fig/gunzip/jan_final.jpg}
    \includegraphics[width=1.5cm]{fig/gunzip/kate_final.jpg}
    \includegraphics[width=1.5cm]{fig/gunzip/joe_final.jpg}
  \end{center}
\end{frame}


\begin{frame}
\frametitle{Our genetic data, online and unzipped}
  \begin{center}
    \includegraphics[width=11cm]{fig/browser_gunzip.png}
  \end{center}
\end{frame}



\begin{frame}
\frametitle{The Foundation for Genomics and Population Health}
  \begin{center}
    \includegraphics[width=11cm]{fig/PHG_banner.png}
  \end{center}
  A useful set of links, papers, documents on genomics and personalized medicine.
\end{frame}




%%%%%%%%%%%%%%%%%%%%%%%%%%%%%%%%%%%%%%%%%%%%%%%%%%%%%%%%%%%%%%%%%%%%%%%%%%%%%
\section{How much medically relevant data can we find in a human genome?}


\begin{frame}
  \frametitle{Genotyping arrays}
  \begin{center}
    \includegraphics[width=3.5cm]{fig/Affymetrix_MouseArray.jpg}
    \includegraphics[width=7cm]{fig/rs11611897_good_SNP.pdf}
  \end{center}
\end{frame}


\begin{frame}
  \frametitle{Genotyping arrays}
  \begin{itemize}
  \item The use of genotyping arrays became really widespread around 2006.
  \item They are still widely used today and cost between \textsterling 50 and \textsterling 400 per person.
  \item This relatively low price made association studies possible.
    \begin{itemize}
    \item Some of these studies require 1,000 of samples.
    \end{itemize}
  \item These arrays are best suited to detect common, rather than rare variants.
  \end{itemize}
\end{frame}


\begin{frame}
  \frametitle{Type 1 diabetes risk}
  \begin{center}
    \includegraphics[width=3.3cm]{fig/VP_T1D.png}
    \includegraphics[width=3.3cm]{fig/DM_T1D.png}
    \includegraphics[width=3.3cm]{fig/LJ_T1D.png}
  \end{center}
  (23AndMe graphs)
\end{frame}



\begin{frame}
  \frametitle{Prostate cancer}
  \begin{center}
    \includegraphics[width=3.3cm]{fig/VP_PCancer.png}
    \includegraphics[width=3.3cm]{fig/DM_PCancer.png}
    \includegraphics[width=3.3cm]{fig/LJ_PCancer.png}
  \end{center}
  (23AndMe graphs)
\end{frame}



\begin{frame}
  \frametitle{Type 2 diabetes}
  \begin{center}
    \includegraphics[width=3.3cm]{fig/VP_T2D.png}
    \includegraphics[width=3.3cm]{fig/DM_T2D.png}
    \includegraphics[width=3.3cm]{fig/LJ_T2D.png}
  \end{center}
  (23AndMe graphs)
\end{frame}



\begin{frame}
  \frametitle{Effects on risk are typically small}
  \begin{center} 
    \includegraphics[width=12cm]{fig/Lshaped.pdf}
  \end{center}
\end{frame}


\begin{frame}
  \frametitle{Effects on risk are typically small}
  \begin{itemize} 
    \item Media often report, with headlines, a risk increase of 50\% for a disease.
    \item But if the baseline is 1/1,000 individuals (i.e. 0.1\%), a 50\% increase means a risk of 0.15\%.
    \item This is, mostly, nothing to be really worried about.
      \begin{itemize}
      \item But many confuse a 50\% increase in risk with a risk of 50\%.
      \item This sort of issues is one of many that motivates some form of genetic counseling.
      \end{itemize}
  \end{itemize}
\end{frame}


\begin{frame}
  \frametitle{More on prediction for inflammatory bowel disease}
  \begin{itemize} 
  \item We could imagine genotyping healthy people and use IBD variants to find a “high risk” group that we can monitor more closely.
  \item Suppose we take people in the top 0.05\% of IBD risk. Even in this high risk group only 1 in 10 people will get IBD.
  \item Even worse, 99\% of real IBD patients WON’T be in this group, and so would be missed by the test!
  \item (Numbers taken from a \href{http://www.genomesunzipped.org/2012/11/dozens-of-new-ibd-genes-but-can-they-predict-disease.php\#more-5226}{recent post} by Luke Jostins at Genomes Unzipped)
  \end{itemize}
\end{frame}


\begin{frame}
  \frametitle{Now testing the cases}
  \begin{itemize} 
  \item How about if we instead introduced a check on a patient’s genome that can be queried by a GP when a patient presents with abdominal pain (one symptom of IBD)?
  \item Let us assume that 10\% of these patients have, in fact, IBD.
  \item We could give some of these patients a confident “nothing to worry about” result ($<$1\% chance of developing IBD) but only 10\% of patients would be told this.
  \item We could give another 7\% of patients a "something to worry about" result (1 in 3 chance of developing IBD).
  \item Still, 83\% of patients will get an inconclusive result, and who knows what any given GP would do with that information.
  \end{itemize}
\end{frame}


\begin{frame}
  \frametitle{The role of the environment should not be understated}
  \begin{center} 
    \includegraphics[width=10cm]{fig/Bach_NEJM_2002.jpg}
  \end{center}
  JF Bach, NEJM 2002
\end{frame}


%%%%%%%%%%%%%%%%%%%%%%%%%%%%%%%%%%%%%%%%%%%%%%%%%%%%%%%%%%%%%%%%%%%%%%%%%%%%%
\section{Some medical implications however}


\begin{frame}
  \frametitle{The more extreme case of Alzheimer's disease}
  \begin{center} 
    \includegraphics[width=10cm]{fig/gunzip_AD.png}
  \end{center}
  From Luke Jostins (\href{http://www.genomesunzipped.org/2011/05/calculating-your-alzheimers-risk.php}{Genomes Unzipped post}) 
\end{frame}



\begin{frame}
  \frametitle{More on the {\it APOE4} alleles}
  \begin{center} 
    \includegraphics[width=6cm]{fig/APOE4_effect.jpg}
  \end{center}
  Plot generated by Nick Eriksson from 23andMe.
\end{frame}




\begin{frame}
  \frametitle{Another rare but informative situation}
  \begin{center} 
    \includegraphics[width=12cm]{fig/G2019S.png}
  \end{center}
\end{frame}


\begin{frame}
  \frametitle{A famous {\it LRRK2}/G2019S patient}
  \begin{center} 
    \includegraphics[width=9cm]{fig/wired_Brin.png}
  \end{center}
  Cover from Wired, June 2010
\end{frame}


\begin{frame}
  \frametitle{Screening for rare variants}
  \begin{center}
    \only<1>{\includegraphics[width=9cm]{fig/VP_carrier.png}}
    \only<2>{\includegraphics[width=9cm]{fig/LJ_carrier.png}}
  \end{center}
\end{frame}





\begin{frame}
  \frametitle{How do consumers react to these results?}
  \begin{itemize}
  \item NEJM, Bloss et al, \href{http://www.nejm.org/doi/full/10.1056/NEJMoa1011893}{Effect of Direct-to-Consumer Genomewide Profiling to Assess Disease Risk}
    \item Survey of randomised or quasi-randomised controlled trials involving adults (aged 18 years and over) in which one group received actual (clinical studies) or imagined (analogue studies) personalised DNA-based disease risk estimates for diseases for which the risk could plausibly be reduced by behavioural change.
  \item The results of this review suggest that communicating DNA-based disease risk estimates has little or no effect on smoking and physical activity.
  \item It may have a small effect on self-reported diet and on intentions to change behaviour. 
  \item Claims that receiving DNA-based test results motivates people to change their behaviour are not supported by evidence. 
  \end{itemize}
\end{frame}




\begin{frame}
  \frametitle{How do consumers react to these results?}
  \begin{itemize}
  \item Marteau et al, \href{http://onlinelibrary.wiley.com/doi/10.1002/14651858.CD007275.pub2/abstract}{Effects of communicating DNA-based disease risk estimates on risk-reducing behaviours}
  \item Primary analyses showed no significant differences between baseline and follow-up in anxiety symptoms ($P=0.8$), dietary fat intake ($P=0.89$), or exercise behaviour ($P=0.61$). 
  \item 90.3\% of subjects who completed follow-up had scores indicating no test-related distress. 
  \item But... they did observe significant associations between composite measures of risk and:
    \begin{itemize}
    \item The total number of screening tests that subjects intended to complete with greater frequency after genetic testing ($P = 0.04$).
    \item As well as test-related distress ($P < 0.001$).
    \end{itemize}
  \end{itemize}
\end{frame}


\begin{frame}
  \frametitle{Exome/whole genome sequencing for Mendelian disorders}
  \begin{center}
    \includegraphics[width=8cm]{fig/Lupski_title.png}\\
    \includegraphics[width=8cm]{fig/Ng1.png}\\
    \includegraphics[width=8cm]{fig/Ng2.png}
  \end{center}
\end{frame}


\begin{frame}
  \frametitle{Rare genetic disorders and sequencing}
  \begin{center}
    \includegraphics[width=9cm]{fig/twins_study.png}
  \end{center}
\end{frame}


\begin{frame}
  \frametitle{Sequencing in a medical context}
  \begin{itemize}
  \item For these rare variants, direct sequencing is in general more useful than genotyping arrays.
  \item There may be considerable medical applications of genome sequencing.
  \item An obvious application is the diagnosis of rare genetic disorders.
  \item But the answer is that we do not yet know how much it matters:
    \begin{itemize}
    \item Growing and related fields are pharmacogenetics and ``personalized medicine''.
    \item Can we tailor drugs to a patient's genetic information?
    \end{itemize}
  \end{itemize}
\end{frame}


\begin{frame}
  \frametitle{More difficult: BBC news, January 9 2009}
  \includegraphics[width=13cm]{fig/bbcnews_BCfreebaby.png}
\end{frame}


\begin{frame}
  \frametitle{Full sequencing is now becoming a reality}
  \begin{center}
    \includegraphics[width=9cm]{fig/nhgri_cost_per_megabase_110221.jpg}
  \end{center}
\end{frame}


\begin{frame}
  \frametitle{Incredibly rapid technological progress}
  \includegraphics[width=3.3cm]{fig/nature_humanGenome.jpg} \ 
  \includegraphics[width=3.3cm]{fig/neanderthal.jpg} \
  \includegraphics[width=3.6cm]{fig/1000genomes_nature.jpg}
  \vspace{1cm}
  Nature 2001, Science 2009 and Nature 2010
\end{frame}


\begin{frame}
   \frametitle{23AndMe is moving into the sequencing business}
   \begin{center}
     \includegraphics[width=8cm]{fig/exome_23AndMe.png}
   \end{center}
   Exome sequencing: target the 1\% protein coding regions of the human genome.
\end{frame}



\begin{frame}
  \frametitle{Should we all get our genome sequenced?}
  \begin{itemize}
  \item The first argument is that is is cheap.
    \begin{itemize}
    \item \$5,000 today, but soon certainly less than \$1,000.
    \end{itemize}
  \item Can we make good use of this information? This is less certain but it should help doctors diagnose some disorders.
  \item This is a IT challenge, and not something the NHS has shown great proficiency for in the past.
  \item I suspect it will happen at some point, the question is when.
  \end{itemize}
\end{frame}





%%%%%%%%%%%%%%%%%%%%%%%%%%%%%%%%%%%%%%%%%%%%%%%%%%%%%%%%%%%%%%%%%%%%%%%%%%%%%
\section{The market for ancestry analysis}

\begin{frame}
  \frametitle{Much of our ancestry is in our genes}
  \begin{center}
    \includegraphics[width=9cm]{fig/europe_PCA.png}
  \end{center}
\end{frame}


\begin{frame}
  \frametitle{Ancestry of the GUNZIP staff}
  \begin{center}
    \only<1>{\includegraphics[width=9cm]{fig/PCA_gunzip_1.png}}
    \only<2>{\includegraphics[width=9cm]{fig/PCA_gunzip_2.png}}
    \only<3>{\includegraphics[width=9cm]{fig/PCA_gunzip_3.png}}
  \end{center}
  See the \href{http://dodecad.blogspot.co.uk/}{Dodecad ancestry blog}.
\end{frame}


\begin{frame}
  \frametitle{Joe Pickrell: am I Jewish?}
  \begin{center}
    \includegraphics[width=5cm]{fig/pickrell_ash_gnz.png}
  \end{center}
  Joe Pickrell (GUNZIP post)
\end{frame}



\begin{frame}
  \frametitle{Separating genome-wide from a few markers}
   \begin{itemize}
   \item There are a few scams in this business.
   \item A genome-wide approach, as sold by 23AndMe, will require 500,000 markers approximately.
   \item But some companies only generate data for a small set of markers (chromosome Y and mitochondrial in particular).
     \begin{itemize}
     \item And then make incredibly precise claims about your ancestry.
     \item These companies provide little to no value and should be avoided.
     \end{itemize}
   \end{itemize}
\end{frame}


\begin{frame}
  \frametitle{An example: some quotes from Britain's DNA founder}
   \begin{itemize}
   \item And what happened with the Britain's DNA project is that about 4 or 5 weeks ago we discovered a remarkable individual, a Mr Ian Kinnaird from Caithness and he has Eve's DNA – he's only two removed from Eve ... he carries a marker called L1b which is only two mutations different from what Eve's marker must have been ... he's Eve's grandson ... 
   \item but we found Sheban DNA –a marker called HV– which we didn’t expect to find, and I say we've got nine people who...
   \item ... the Bible, through the Britain's DNA project and other research is really beginning to come alive...
   \end{itemize}
\end{frame}

\begin{frame}
  \frametitle{Other applications: finding biological parents and forensics}
  \begin{itemize}
  \item DNA is well known for its ability in a legal context.
  \item Another application is the identification of related individuals:
    \begin{itemize}
    \item Most DTC companies can put you in contact with individuals that are distantly related.
    \item As mentioned above, some people learned that their parents are not biological parents.
    \item Children born from sperm donation could find their biological parent.
    \end{itemize}
  \end{itemize}
\end{frame}




\begin{frame}
  \frametitle{Summary}
  \begin{itemize}
  \item DTC genetics is a new market, which has not yet matured.
  \item There is considerable uncertainty on the legal status of these companies.
    \begin{itemize}
    \item Do they provide a medical test?
    \end{itemize}
  \item Impact on customers seems limited, but not insignificant.
  \item The market for ancestry estimation is a more difficult one, and many claims are overstated.
  \item Limited impact on human health of most findings make the debate somewhat marginal at this stage, but the advent of high throughput sequencing studies is changing the situation.
  \end{itemize}
\end{frame}

\end{document}
